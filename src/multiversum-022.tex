% !TeX TS-program = xelatex
% !TEX root = main.tex

% How to:
%


% Load class with all definitions. Do not remove this line.
% Options will be passed to Memoir
\documentclass[final]{multiversum}
%

%%%
% Set those variables

% Authors of the document.
% e.g. Max Mustermann, Erika Musterfrau
\multiauthor{Jan, Henri, Yoann, Hanna, Richard}

% Date of release.
% e.g. 31.12.2074
\multidate{Mai 2020}

% Number of release, no leading zeros.
% e.g. 15
\multiausgabe{22}

% Losung
% e.g. Die Kuh lief um den Teig.
\multilosung{Mein Schiff hei{\ss}t Nieselspatz.}


% Logo
% Use a different logo. Defaults to Ueberschrift.svg
%\multilogo{Ueberschrift_xmas}

%
%%%

%%%%%%%%%%%%%%%%%%%%%%%%%%%%%%%%% DOCUMENT %%%%%%%%%%%%%%%%%%%%%%%%%%%%%%%%%
\begin{document}

\makemultititle
%

% PUT BODY HERE
\section{Was bisher geschah...}

\subsection{Schiffstaufe}
Das Problem der spontanen unbenannten NPCs ist ein Bekanntes und die Achillesferse eines jeden Spielleiters.
Es gibt einen neuen NPC, alle Spieler schauen erwartungsvoll zum Spielleiter und fragen die Ingame-Frage des Grauens:

\enquote{Wie heißt du?}

Die klare Spielleiterantwort kommt promt:

\enquote{Äh, \textit{ich}, äh, \textit{heiße}, äh, warum lädt das denn nicht, äh, \textit{Cendgate}.}

Klar, das klingt wie eine Zahnpastamarke, aber der Namensgenerator hat das halt ausgespuckt.
Leider haben sich das nun alle Spieler gemerkt und der Spielleiter tut gut daran, sich den Namen schleunigst aufzuschreiben.
Mit an Sicherheit grenzender Wahrscheinlichkeit wird genau dieser unausgearbeitete NPC der dauerhafte Begleiter der Gruppe.

Die meisten Spielleiter wissen inzwischen, welche Namensgeneratoren zuverlässig sind und einigermaßen akzeptable Namen ausspucken.
Das gleiche Problem habe ich aktuell in meiner Runde, wo dauernd neue Schiffsnamen her müssen.
Leider sind die Schiffe in diesem speziellen Abenteuer zum Verwechseln ähnlich benannt.
Da gibt es die Seeadler, die Sturmvogel, die\dots{ }nochmal Seeadler, diesmal \enquote{Seeadler von Beilunk}.

Da zeichnet sich ein Muster ab, irgendetwas mit Wetter oder See und einem Vogel.
Nachdem die Gruppe einen Hilferuf an ein neues, eigenbenanntes Schiff, die Sonnenmöwe, abgesetzt hat, bin ich nun auf weiteres vorbereitet.

Wenn *du* genau das gleiche spezielle Problem hast, dann nutze doch den Würfelgenerator auf der Rückseite!
Dort findest du neben diesem Nischenproblem auch wunderbare Lösungen weitere sehr eingeschränkte Anwendungen!
\verfasser{Hanna}
\pagebreak

\begin{table*}[!t]
Erschaffe dein perfektes Schiff, indem auf zwei Rubriken würfelst, eine aus der ersten und eine aus der zweiten Spalte.
Du musst die Wörter nur noch aneinander hängen.
Bei Bedarf oder zufälliger Dopplung kannst du noch optionale Zusätze aus den Rubriken der dritten Spalte hinzufügen.

\begin{framed}
\begin{tabular}{p{0.05\textwidth}p{0.2\textwidth}p{0.05\textwidth}p{0.25\textwidth}p{0.05\textwidth}p{0.3\textwidth}}
\textbf{1W11} & \textbf{Wetter}                & \textbf{1W23}   & \textbf{Vögel}         & \textbf{1W11}   & \textbf{Suffixe}       \\
1             & Donner                         & 1               & Adler                  & 1               & von Beilunk            \\
2             & Sturm                          & 2               & Königspinguinplüsch    & 2               & der Meere              \\
3             & Regen                          & 3               & Storch                 & 3               & des Stolzes            \\
4             & Schnee                         & 4               & Spatz                  & 4               & der Rose               \\
5             & Niesel                         & 5               & Krähe                  & 5               & für Elise              \\
6             & Sonne(n)                       & 6               & Meise                  & 6               & des Verderbens         \\
7             & Hagel                          & 7               & Möwe                   & 7               & meines Großvaters      \\
8             & Bewölkte(r)                    & 8               & Specht                 & 8               & +2                     \\
9             & Düster                         & 9               & Drossel                & 9               & Inc.                   \\
10            & Aufklärende(r)                 & 10              & Ente                   & 10              & des Todes              \\
11            & Blitz                          & 11              & Falke                  & 11              & *ohne Rettungsboote    \\
\textbf{1W9} & \textbf{Wasser}                 & 12              & Pfau                   & \textbf{1W21}   & \textbf{Präfixe}       \\
1             & See                            & 13              & Potoo                  & 1               & Flotte(r)              \\
2             & Meeres                         & 14              & Schwan                 & 2               & Majestätische(r)       \\
3             & Wasser                         & 15              & Schwalbe               & 3               & Scharlachrote(r)       \\
4             & Teich                          & 16              & Rabe                   & 4               & Starke(r)              \\
5             & Tropfen                        & 17              & Vogel                  & 5               & Unbeugsame(r)          \\
6             & Wellen                         & 18              & Taube                  & 6               & Lecker                 \\
7             & Piss                           & 19              & Sperling               & 7               & Kecke(r)               \\
8             & Flie{\ss}ende(r)               & 20              & Zaunkönig              & 8               & Prüde(r)               \\
9             & Wogende(r)                     & 21              & Eisvogel               & 9               & Infame(r)              \\
\textbf{1W11} & \textbf{Erz}                   & 22              & Neuntöter              & 10              & Hufloser(r)            \\
1             & Silber                         & 23              & Drache                 & 11              & Ehrlose(r)             \\
2             & Bronze                         & \textbf{1W9}    & \textbf{Säugetiere}    & 12              & Hirtenlose(r)          \\
3             & Blech                          & 1               & Robbe                  & 13              & Blaue(r)               \\
4             & Kupfer                         & 2               & Schnabeltier           & 14              & Intakte(r)             \\
5             & Chrom             	           & 3               & Hamster                & 15              & Geblümte(r)            \\
6             & Eisen                          & 4               & Stier                  & 16              & Sweet, sweet           \\
7             & Stahl                          & 5               & Pferd                  & 17              & Alte(r)                \\
8             & Gold                           & 6               & Otter                  & 18              & Neue(r)                \\
9             & Stein                          & 7               & Löwe                   & 19              & Niet- und Nagelfeste(r)\\
10            & Kreide                         & 8               & Hirsch                 & 20              & Nein, die(der) andere  \\
11            & Kristall                       & 9               & Bär                    & 21              & Vermaledeite(r)        \\
\end{tabular}
\end{framed}
\end{table*}


\section{An einem anderen Ort}

\subsection{An den Befehlshaber der Perlenmeerflotille vor Boran}
Eure Exzellenz,\\[1em]
wir schreiben Euch, weil wir in größter Not sind.
Mein geneigter Reisebegleiter und ich sind unterwegs zu einer Expedition in den Norden der Insel Maraskan, um einen großen Nodex der neuen Achaz aus der letzten oder vorletzten Epoche zu finden.
Nun benötigten wir ein Schiff zur Überfahrt und haben einen Kapitän gefunden, der bereit war uns mitzunehmen.
Er trug den Namen Kodnas Han und wir fanden nur wenig später einen Steckbrief, der ihn als einen bösartigen Piraten auswies.
Am gleichen Tag erreichte uns noch eine Nachricht, dass der Kapitän die Absicht hat, uns gefangen zu nehmen und zu verkaufen.
Da es für uns jedoch dringend notwendig ist, unsere Expedition fortzusetzen, befinden wir uns in einem Dilemma.\\
Wir ersuchen Euch daher um Eure Hilfe.
Morgen früh werden wir mit unserer Eskorte auf dem Schiff auslaufen und bitten Euch, das Schiff zu diesem Zeitpunkt abzufangen und den Piraten festzusetzen.
Bei dem Schiff handelt es sich um eine Thalukke, die zur dritten Stunde des Tages den Hafen verlassen wird.
Wir sind dringend auf Eure Hilfe angewiesen und unsere körperliche Unversehrtheit ist in großer Gefahr.\\[1em]
Hochachtungsvoll,
\begin{flushright}
    \textit{Magus Abdul el Mazar, emeritierter Magister Ordinarius des Lehrstuhls für Beschwörung an der Akademie von Pentagramma, Hexagramma und Heptagramma zur Meisterung jenseitiger Entitäten zu Rashdul}
\end{flushright}
\verfasser[DSA 4.1]{Henri, Yoann \& Jan}

\subsection{Anzeige im Eiskunstlauf-Forum}
Gerade in Abu Dhabi angekommen. 
Super nicer Icering, aber sonst mega hot hier.
Glaube ich lasse beim Wettbewerb die Strumpfhosen weg ;-)
Denke darüber nach, Sandalen-Schlittschuhe zu kaufen.
Was sind eure Ehrfahrungen damit?
Yey oder Ney?
\verfasser[Shadowrun]{Hanna}

\subsection{Neulich in der Taverne}
Der Barde:\\
"Das Dichten ist mir einerlei,\\
Doch hau'n die Söldner euch zu Brei!"\\[1em]
Der Erzähler:\\ 
"Der Barde war wohl nicht ganz helle,\\
Denn kaum gesagt flog blitzeschnelle\\
Ein Pfeil in seinen Hals und blieb.\\
Den ersten Söldner aber hieb'\\
Der Eichbald mit dem Stabe um. \\
Der Zweite schaute reichlich dumm,\\
Denn der wollte zu Rodrunk hin \\
Und fand ne Axt in seinem Kinn.\\
Ein Schildstoß stoppt den Dritten bald\\
Und Madalin, die macht ihn kalt.\\
Der Barde schaut nur voller Graus, \\
Die Heldengruppe lacht ihn aus.\\
Und ohne Luft fällt er danieder,\\
Von diesem hörte man nie wieder."
\begin{flushright}- Logram groscho Gorgrom, Spalter von Kinnen\end{flushright}
\verfasser[DSA 4.1]{Richard}

% \section{Werbung}
% Wollten sie schon immer mal Boran anzünden? Dann kaufen Sie jetzt\\
% \begin{center}\textsc{BORANZÜNDER}!\\\end{center}
% Die patentierten Brandbeschleuniger vermögen es sogar das älteste Maraskaner Hartholz zu entfachen\footnote{Im Verhältnis von mindestens 2:1.}!

% Kaufen Sie sich jetzt\\
% \begin{center}\textsc{BORANZÜNDER}\\\end{center}
% für ihren Kamin und ihre Feuerstelle. 
% \verfasser[DSA 4.1]{Yoann}

\begin{termine}
% Put dates here:
\item Monatstreffen: 16.05.2021, 19 Uhr
\item Rollenspielabend mit der Fachschaft I/1:\\ 08.05.2021, 18 Uhr
\end{termine}
\impressum

\end{document}
%%%%%%%%%%%%%%%%%%%%%%%%%%%%%%%%% END DOCUMENT %%%%%%%%%%%%%%%%%%%%%%%%%%%%%%%%%
