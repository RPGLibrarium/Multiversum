% !TeX TS-program = xelatex
% !TEX root = main.tex

% How to:
%


% Load class with all definitions. Do not remove this line.
% Options will be passed to Memoir
\documentclass[final]{multiversum}
%

%%%
% Set those variables

% Authors of the document.
% e.g. Max Mustermann, Erika Musterfrau
\multiauthor{Yoann}

% Date of release.
% e.g. 31.12.2074
\multidate{176.02.2020}

% Number of release, no leading zeros.
% e.g. 15
\multiausgabe{19}

% Losung
% e.g. Die Kuh lief um den Teig.
\multilosung{Es heißt \enquote{Multivera}: Substantivertes Adjektiv Nominativ Plural Neutrum!}


% Logo
% Use a different logo. Defaults to Ueberschrift.svg
%\multilogo{Ueberschrift_xmas}

%
%%%

%%%%%%%%%%%%%%%%%%%%%%%%%%%%%%%%% DOCUMENT %%%%%%%%%%%%%%%%%%%%%%%%%%%%%%%%%
\begin{document}

\makemultititle
%

\section{Aus der Redaktion}
\subsection{Multiversum wurde auf Eis gelegt}
Liebe Leser*innen von heute, 

Die Winterpause ist endlich vorbei! Auch dieses Jahr lassen wir unsere unermüdlichen Schreiberlinge auf dem Pergament tanzen um Euch aus Raum und Zeit zu berichten. 

Die erste erfreuliche Nachricht erreichte uns unmittelbar aus dem Großen Archiv\footnote{\url{https://archiveprogram.github.com/}}. Über die letzten Montate hat sich das Multiversum zu einem stolzen und integralen Teil der menschlichen Geschichtsschreibung entwickelt. Die hiesigen Archivare haben beschlossen das Multiversum in \textit{Das Backup} aufzunehmen und Euer geliebtes Flugblatt auf Eis gelegt. 
Das Multiversum ist für alle Ewigkeit\footnote{die Technik verspricht circa 150 Jahre} bewahrt. Auch eure Enkeln werden noch aus diesen literarischen Meisterwerke Weisheiten schöpfen können.

Wir danken Euch, liebe Leserinnen und Leser, für die treue Leserschaft und wünschen Euch auch bei dieser neuen Ausgabe eine erholsame Lektüre.

\vspace{1em}\noindent
Liebe Archäolog*innen und Assistent*innen von morgen,

Danke, dass Sie sich die Mühe gemacht habe die alten Filmbänder zu enteisen\footnote{oder abzutropfen}, die Schnipsel wieder zusammen zu kleben, und die QR-Codes zu entschlüssen.
Es war bestimmt nicht einfach das sonderbare LaTeX-Format\footnote{Zugegeben, es war auch 2020 bereits ein Relikt aus vergangenen Zeiten. Wir hatten nur noch keine besseren Ideen für Textsatz.} zu verstehen und die vielen Ausgaben zu kompilieren\footnote{ Das heißt soviel, wie \enquote{übersetzen}; ein komisches Wort, ich weiß.}.
Leser aus der Zukunft, gönnen Sie sich einige Minuten Erholung, brauen Sie sich eine heiße Tasse (Soja-)Kaffee\footnote{anregendes Heißgetränk}, lehnen Sie sich zurück und genießt auch Sie diese neue alte Ausgabe des Multiversums.
Wir wünschen Ihnen in der finsteren\footnote{ Klimakatastrophe, Ressourcenknappheit, Weltuntergang. Sie wissen schon.} Zukunft alles Gute und hoffen Ihnen zumindest ein kleines Lächeln auf den Lippen\footnote{Kommunikationsorgane} zu hinterlassen. 
\verfasser{Eure Redaktion}

\section{An einem anderen Ort}

\subsection{Titel 1}
Lorem Ipsum
\verfasser{Autor2}

\subsection{Titel 2}
Lorem Ipsum, Text endet an Zeilenende.
\Verfasser[System]{Autor2}


\begin{termine}
% Put dates here:
\item Termin: Termin: DD.MM.YY, hh Uhr
  \item Termin: DD.MM.YY - DD.MM.YY
\end{termine}
\impressum

\end{document}
%%%%%%%%%%%%%%%%%%%%%%%%%%%%%%%%% END DOCUMENT %%%%%%%%%%%%%%%%%%%%%%%%%%%%%%%%%
