% !TeX TS-program = xelatex
% !TEX root = main.tex

% How to:
%


% Load class with all definitions. Do not remove this line.
% Options will be passed to Memoir
\documentclass[final]{multiversum}
%
\usepackage{fontspec}
\newfontfamily\khmerfont{Noto Serif Khmer ExtraLight}

%%%
% Set those variables

% Authors of the document.
% e.g. Max Mustermann, Erika Musterfrau
\multiauthor{Yoann}

% Date of release.
% e.g. 31.12.2074
\multidate{176.02.2020}

% Number of release, no leading zeros.
% e.g. 15
\multiausgabe{19}

% Losung
% e.g. Die Kuh lief um den Teig.
\multilosung{Es heißt \enquote{Multivera}: Substantivertes Adjektiv Nominativ Plural Neutrum!}


% Logo
% Use a different logo. Defaults to Ueberschrift.svg
%\multilogo{Ueberschrift_xmas}

%
%%%

%%%%%%%%%%%%%%%%%%%%%%%%%%%%%%%%% DOCUMENT %%%%%%%%%%%%%%%%%%%%%%%%%%%%%%%%%
\begin{document}

\makemultititle
%

\section{Aus der Redaktion}
\subsection{Multiversum wurde auf Eis gelegt}
Liebe Leser*innen von heute, 

Die Winterpause ist endlich vorbei! Auch dieses Jahr lassen wir unsere unermüdlichen Schreiberlinge auf dem Pergament tanzen um Euch aus Raum und Zeit zu berichten. 

Die erste erfreuliche Nachricht erreichte uns unmittelbar aus dem Großen Archiv\footnote{\url{https://archiveprogram.github.com/}}. Über die letzten Montate hat sich das Multiversum zu einem stolzen und integralen Teil der menschlichen Geschichtsschreibung entwickelt. Die hiesigen Archivare haben beschlossen das Multiversum in \textit{Das Backup} aufzunehmen und Euer geliebtes Flugblatt auf Eis gelegt. 
Das Multiversum ist für alle Ewigkeit\footnote{die Technik verspricht circa 150 Jahre} bewahrt. Auch eure Enkeln werden noch aus diesen literarischen Meisterwerke Weisheiten schöpfen können.

Wir danken Euch, liebe Leserinnen und Leser, für die treue Leserschaft und wünschen Euch auch bei dieser neuen Ausgabe eine erholsame Lektüre.

\vspace{1em}\noindent
Liebe Archäolog*innen und Assistent*innen von morgen,

Danke, dass Sie sich die Mühe gemacht habe die alten Filmbänder zu enteisen\footnote{oder abzutropfen}, die Schnipsel wieder zusammen zu kleben, und die QR-Codes zu entschlüssen.
Es war bestimmt nicht einfach das sonderbare LaTeX-Format\footnote{Zugegeben, es war auch 2020 bereits ein Relikt aus vergangenen Zeiten. Wir hatten nur noch keine besseren Ideen für Textsatz.} zu verstehen und die vielen Ausgaben zu kompilieren\footnote{ Das heißt soviel, wie \enquote{übersetzen}; ein komisches Wort, ich weiß.}.
Leser aus der Zukunft, gönnen Sie sich einige Minuten Erholung, brauen Sie sich eine heiße Tasse (Soja-)Kaffee\footnote{anregendes Heißgetränk}, lehnen Sie sich zurück und genießt auch Sie diese neue alte Ausgabe des Multiversums.
Wir wünschen Ihnen in der finsteren\footnote{ Klimakatastrophe, Ressourcenknappheit, Weltuntergang. Sie wissen schon.} Zukunft alles Gute und hoffen Ihnen zumindest ein kleines Lächeln auf den Lippen\footnote{Kommunikationsorgane} zu hinterlassen. 
\verfasser{Eure Redaktion}

\section{An einem anderen Ort}
\subsection{Mix \& Match - Schmuggleredition} 
Die \textit{Multiversums-Experten für diskrete Logistik} präsentieren die Top 12 Geheimverstecke für alles, was vor dem Gesetz besser verborgen bleiben sollte. Was verbirgt sich wo? Mix \& Match!

\begin{framed}
\centering
\begin{tabular}{>{\bfseries}p{0.01\textwidth}p{0.9\textwidth}}
\multicolumn{2}{c}{\textbf{1W12}}\\
    1 & Ein Lichtschwert \\
    2 & Ein kleiner, aber explosiver Prototyp \\
    3 & Ein wirklich böses Holocron der Sith \\
    4 & Kiloweise Kokain \\
    5 & Der Kopf eines Wookiees \\
    6 & Die letzten 100 Gramm Tigermohn der Stadt \\
    7 & Ein gratis BTL-Chip zur Kostprobe \\
    8 & Der +2-Charisma-Ring eines hässlichen Flusspiraten \\
    9 & Eine Enterhakenkanone ohne Lizenz \\
    10 & Eine Monofilamentpeitsche \\
    11 & Ein versteckter Schmuggler \\
    12 & Das Juwel von Yavin \\
\end{tabular}
    \vspace{1em}

\begin{tabular}{>{\bfseries}p{0.01\textwidth}p{0.9\textwidth}}
\multicolumn{2}{c}{\textbf{1W12}}\\
    1 & Im Inneren eines Medizindroiden \\
    2 & In einer Raviolidose \\
    3 & In einer tiefen Manteltasche zwischen Wüstensand und Schokolade \\
    4 & In einer Bananenkiste unter den Bananen \\
    5 & Im Kühlschrank (neben dem Joghurt) \\
    6 & In einer Zunderdose \\
    7 & Hinter den Fußleisten bei der Ikea-Kommode \\
    8 & In einem tiefen Dekolleté \\
    9 & Im Kofferraum eines polizeilich beschlagnahmten Autos \\
    10 & Im Inneren einer kybernetischen Hand \\
    11 & Im Schmuggelversteck eines Raumschiffs \\
    12 & In einem großen Müllcontainer in einer dunklen Gasse \\
\end{tabular}
\end{framed}
\verfasser[Beliebig]{Franca}

\subsection{66.6 MHz - Radio Inferno}
Good Morning, Barovia! \textit{For those about to rock – we salute you}! This is Radio Inferno, I’m your eldritch host, and we are \textit{Back In Black} and live on-air. Wanna listen to your favourite rock songs? Just call the \textit{Number of the Beast} and I‘ll play your \textit{Symphony of Destruction}.

In for an interview with Radio Inferno is Strahd of Zarovich, vampire overlord of Barovia. Glad to have you here, Strahd, Radio Inferno certainly has a \textit{Sympathy For The Devil}, which happens to be your nickname in Barovia. But before we talk about you being \textit{Wanted Dead Or Alive}, it’s time for the Barovian news…

\centerline { +++ }

\textit{Smoke on the Water}: Man with axe tries to set lake on fire by using an axe.

\textit{Thunderstruck}: Tempest cleric tries to \textit{Ride The Lightning}. No severe injuries, Cleric still \textit{Living on A Prayer}.

\textit{Dirty Deeds Done Dirt Cheap}: Nobleman magically vanishes bloods stains for free after \textit{Sabbath, Bloody Sabbath}.

\textit{Slayer}: Woman catches knife in mid-air, thereby killing vampire son of \textit{Judas Priest}.

\textit{Paranoid}: Lack of darkvision can lead to \textit{Fear of The Dark}.

\textit{Bat out of Hell}: \textit{The Highway to Hell} is closed in both directions due to a motorcycle accident caused by bats. Please \textit{Run To the Hills} or use the \textit{Stairway to Heaven}.

\centerline { +++ }

\textit{Welcome Home} to the studio, Barovia! Strahd, \textit{Heaven and Hell} are listening and \textit{If you Want Blood, You’ve got it}! So, now that you’ve got your drink and I’ve got your attention – Barovia has tons of questions. Tell me, Strahd: What's it like to be the \textit{Master of Puppets}? […]

\verfasser[sehr frei nach D\&D 5]{Franca}

\subsection{Auschnitt aus einem Tagebuch}
\textit{19. Aug. 2074} Ich habe nun endlich mit FO31 angefangen. Es ist noch nicht lebensfähig. Meine Zeit ist knapp.  Protheus nimmt mich sehr in Anspruch. 

\noindent\textit{4. Dez. 2074} FO31 ist verstorben. Nach so lange Zeit ist er eingegangen. Diesmal beginne ich mit der Konstruktion des Abdomen. Die Welt der Cybertechnologie erschließt sich mir noch nicht vollständig, aber die Matrix ist voller Hilfe. Es wird dauern, bis ich mich an die anderen Spielregeln gewöhnt habe.

\noindent\textit{10. Dez. 2074} Bei Cyberkonstruktionen mache ich noch sehr viele Fehler. Biologisches Gewebe verzeiht mehr. Ich muss weiter lernen und besser werden, wenn ich diesen Versuch zu einem Erfolg führen möchte.

\noindent\textit{24. Dez. 2074} [durchgestrichen]

\noindent\textit{20. Jan. 2075} Ich habe einen neuen Assistenten gewonnen, dessen Wissen in Cyberimplantierung sicherlich zuträglich sein wird.

\noindent\textit{13. Mai 2075} Habe mich in Toxine eingelesen. Sehr spannendes Thema. FO31 wird davon profitieren. Anonsten Fortschritt langsam aber stetig. 

\noindent\textit{9. Nov. 2076} Er lebt! Die jahrelange Arbeit war erfolgreich.
\Verfasser[SR 5]{Yoann}

\subsection{Werbung}
Kommt in die Krabbenstadt, denn nur dort  gibt es Ungeheuer-Bier!
Allein hier bekommt ihr den Geschmack nach Meer und mit jedem Schluck das fesselnde Gefühl von Saugnäpfen an euren Gaumen.

Dazu gibt es nun die neuen Geschmacksrichtungen Waldmeister und Alge!\footnote{Leider ist unser Azubi beim Etikettieren eingeschlafen. Wir wissen nicht was wir da ausschenken, Geschmacksrichtungen für echte Abenteurer!}
Manch einer fragt sich: Wie entsteht eigentlich der Tentakeleffekt?
Hierfür wird der beste Vorleser des Landes vor dem Reifeprozess neben den Bierkessel gesetzt.
Er liest dem Bier solange Schauergeschichten vor, bis es glaubt es sei ein Meer voller tentakliger Meeresungeheuer.\footnote{Dies schafft nur ein wahnsinniger Literat oder literarischer Wahnsinniger.}
Danach wird das Gebräu einsam und allein in einem dunklen Raum seinem Reifeprozess überlassen.

Unser Bier ist besonders auf Sauberkeit bedacht. Keine Schädlinge können es besiedeln.\footnote{Man möchte sich nicht vorstellen, was mit lebenden Organismen geschieht, wenn dieser Effekt bereits bei einer einfachen Flüssigkeit auftritt.}Daher ist durch unser Bier im Umland die Redewendung ,,So sicher wie Ungeheuer-Bier!'' entstanden. Kosten auch SIE einen reinen Schluck Kultur.
\verfasser[Turbo Fate]{Sabine} 


\subsection{Und nun das Wetter.} 
Es berichtet Professor Eulenbach vom Korani-Planck-Institut.

Ab Dienstag erwarten uns vermehrt Unwetter. Eine Wand temporaler Anomalien zieht von Osten her auf und erstreckt sich im Laufe der Woche über das gesamte Bundesgebiet. Das Bundesamt für Astrahlenschutz warnt vor eröhten entropischen Werten zwischen 7 und 9 µ{\khmerfont៚}. Sensible Gemüter mit Chronophobie sollen sich auf die nächsten Tage vorbereiten und falls möglich zu Hause bleiben.

Ab Freitag erwartet uns dann ein sinuales Tief, welches die temporalen Anomalien langsam verdrängt und zum Wochenende schließlich thaumaturgische Böhen mit sich bringt. 

Dachdem in den letzten Jahren immer wieder Rekordwerte bei multidimensionalen Phänomenen verzeichnet wurden, rechnen Experten diese Sommer mit einem neuen Rekordtief seit dem Begin der Wetteraufzeichnungen. 
\Verfasser{Richard \& Yoann}

\begin{termine}
% Put dates here:
\item Termin: Termin: DD.MM.YY, hh Uhr
  \item Termin: DD.MM.YY - DD.MM.YY
\end{termine}
\impressum

\end{document}
%%%%%%%%%%%%%%%%%%%%%%%%%%%%%%%%% END DOCUMENT %%%%%%%%%%%%%%%%%%%%%%%%%%%%%%%%%
