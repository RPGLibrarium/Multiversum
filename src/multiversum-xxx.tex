% !TeX TS-program = xelatex
% !TEX root = main.tex

% How to:
%


% Load class with all definitions. Do not remove this line.
% Options will be passed to Memoir
\documentclass[final]{multiversum}
%

%%%
% Set those variables

% Authors of the document.
% e.g. Max Mustermann, Erika Musterfrau
\multiauthor{}

% Date of release.
% e.g. 31.12.2074
\multidate{}

% Number of release, no leading zeros.
% e.g. 15
\multiausgabe{}

% Losung
% e.g. Die Kuh lief um den Teig.
\multilosung{}


% Logo
% Use a different logo. Defaults to Ueberschrift.svg
%\multilogo{Ueberschrift_xmas}

%
%%%

%%%%%%%%%%%%%%%%%%%%%%%%%%%%%%%%% DOCUMENT %%%%%%%%%%%%%%%%%%%%%%%%%%%%%%%%%
\begin{document}

\makemultititle
%

% PUT BODY HERE
\section{Was bisher geschah...}

\subsection{Workshop: ABC}
Lorem Ipsum
\verfasser{Autor1}

\subsection{In der Redaktion}
Lorem Ipsum
\verfasser{Redaktion}

\section{An einem anderen Ort}

\subsection{Titel 1}
Lorem Ipsum
\verfasser{Autor2}

\subsection{Titel 2}
Lorem Ipsum, Text endet an Zeilenende.
\Verfasser[System]{Autor2}


\begin{termine}
% Put dates here:
\item Termin: Termin: DD.MM.YY, hh Uhr
  \item Termin: DD.MM.YY - DD.MM.YY
\end{termine}
\impressum

\end{document}
%%%%%%%%%%%%%%%%%%%%%%%%%%%%%%%%% END DOCUMENT %%%%%%%%%%%%%%%%%%%%%%%%%%%%%%%%%
