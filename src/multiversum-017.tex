% !TeX TS-program = xelatex
% !TEX root = main.tex

% How to:
%


% Load class with all definitions. Do not remove this line.
% Options will be passed to Memoir
\documentclass[final]{multiversum}
%

%%%
% Set those variables

% Authors of the document.
% e.g. Max Mustermann, Erika Musterfrau
\multiauthor{Yoann, Franca, Maria}

% Date of release.
% e.g. 31.12.2074
\multidate{Dezember 2019}

% Number of release, no leading zeros.
% e.g. 15
\multiausgabe{17}

% Losung
% e.g. Die Kuh lief um den Teig.
\multilosung{In diesem Falle, Riesenqualle}

% Xmas logo
\multilogo{Ueberschrift-Xmas2019}
%
%%%

%%%%%%%%%%%%%%%%%%%%%%%%%%%%%%%%% DOCUMENT %%%%%%%%%%%%%%%%%%%%%%%%%%%%%%%%%
\begin{document}

\makemultititle
%

% PUT BODY HERE
\section{Aus der Redaktion}

\subsection{Spendenkampagne 2019}
Liebe Leserinnen und Leser, bitte verzeihen Sie die Störung. Es ist uns ein bisschen unangenehm, aber kommen wir gleich zur Sache. An diesem Tage sind Sie im Multiversum gefragt, um die Unabhängigkeit des RPG Librariums Aachen e.V. zu sichern:

Heute ist der 63. Tag unserer Bücherkaufkampagne. Der Librarium bekommt im Jahr durchschnittlich 3,5 Bücherwünsche, aber 99\% der Leserinnen und Leser schlagen nichts vor. Wenn alle Leser, die jetzt mitlesen, einen Bücherwunsch einsenden würden, wäre unsere Kampagne im Nu abgeschlossen.
Wir sind kein kommerzieller Verein, sondern ein Ort der Kunst, Kultur und natürlich der Ausgaben. Verschwenderische Ausgaben haben jedoch in unserem Verein nichts zu suchen. Im Gegensatz zu anderen großen Bibliotheken haben wir keine Personalkosten und auch unsere Hosting- und Verwaltungsgebühren belaufen sich nur auf einen Bruchteil unserer Einnahmen. Deshalb benötigen wir Ihre Hilfe. Um es dem Finanzamt recht zu machen, darf der Librarium auch in diesem Jahr keinen zu hohen Gewinn erwirtschaften.

Es ist leicht, diese Nachricht zu ignorieren und die meisten werden das wohl tun. Wenn Sie den Librarium nützlich finden, nehmen Sie sich an diesem Wochentag bitte eine Minute Zeit und geben dem Verein mit Ihrer Bücherwunschspende etwas zurück, damit der Librarium weiter bestehen kann. Vielen Dank!
\Verfasser{Redaktion}

\section{An einem anderen Ort}

\subsection{Clan Eisenfaust informiert}

\setsubsubsecheadstyle{\centering\bfseries}
\subsubsection{Der Clan Eisenfaust begrüßt erfreut:}

\begin{center}
   Thif Eisenfaust.
\end{center}
\vspace{-0.5em}
Unter mysteriösen Umständen auferstanden von den Toten. Schön dich wieder zu haben!
\Verfasser[D\&D 5]{Maria}

\subsection{Strickschulden sind Ehrenschulden}
\enquote{So, \enquote{Bei Mittlerer Hitze 20 Minuten goldbraun backen.} Was soll das denn schon wieder heißen? Ach was soll's, 180°C werden schon passen.
Jetzt noch die Zeitschaltuhr\dots}

Erna Brügge wollte gerade nach der Eieruhr greifen, als plötzlich die Eingangstür zu ihrer kleinen Einzimmerwohnung mit einem lauten Knall explodierte. So schnell, wie es ihr gebrechlicher Körper erlaubte, wandte sie sich um. Durch die Trümmer der Haustür zwängte sich der mächtige Körper eines Trolls.

\enquote{Hallo Erna.}, begrüßte er sie mit rauchiger Stimme.
\enquote{Zweimal habe ich dir schon Aufschub gewährt. Jetzt ist es Zeitpunkt gekommen, an dem du lieferst.}

Mit einem panischen Blick in den Augen tat Erna einen Schritt zurück und stieß mit ihrem Rücken gegen die warm werdende Ofentür. \enquote{Ich hab sie noch nicht fertig!}, versuchte sie mit zittriger Stimme den Eindringling zu besänftigen. \enquote{Bitte gib mir noch was Zeit! Ich brauche nur eine weitere Woche\dots}

Jäh wurde sie von dem grollenden Troll unterbrochen.
\enquote{Nein! Ich will sie jetzt. Ich habe dir Essen gegeben, als du am Verhungern warst. Ich habe dir Unterschlupf gewährt, als es geregnet hat und ich habe dir Munition gegeben, als du deine letzte Patrone verschossen hattest. Warum?} Der Troll suchte nach den richtigen Worten. \enquote{Ich kann es dir nicht sagen. Wahrscheinlich aus Mitleid.}

Erna suchte verzweifelt einen Weg zum Fenster, durch das sie über die Feuerwehrleiter entkommen könnte, doch die breiten Schultern versperrten ihr jegliche Fluchtmöglichkeiten.
\enquote{Doch jetzt will ich endlich das, was du mir damals als Bezahlung versprochen hattest: Meine gestrickten Wollsocken.} Sein linkes Auge zuckte ein wenig. \enquote{Blaue kuschelige warme Wollsocken aus Merinowolle\dots}
Erna verlor die Kraft in ihren Beinen und rutschte an der Ofentür zu Boden.

\enquote{Bitte, nur zwei Tage. Übermorgen kann ich liefern.}

\enquote{Deine Strickschulden sind dein Problem. Such dir jemand Anderes, dem du deine Lügen verkaufen kannst, aber jetzt will ich endlich, was mir zusteht!}
\Verfasser[SR 5]{Yoann}

\subsection{Es begab sich aber zu der Zeit …}
... dass ein Auftrag von der Yakuza ausging, dass eine Lieferung Better-Than-Life-Chips nach Berlin zu schmuggeln sei.
Und diese Lieferung war nicht die erste und sie geschah im Jahre 2077, als Lofwyr CEO in Neu-Essen war.
Doch jeder Runner ging, da Heiligabend war, ein jeder in seine Stadt, und keiner wollte nach Berlin fahren.

Da machte sich ein Kurier auf aus dem Rhein-Ruhr-Plex, aus der Stadt Köln, durchs Rheinland zur Stadt der Yakuza, die da heißt Düsseldorf, weil er ein zuverlässiger Runner der Yakuza war und außerdem noch keine Pläne für Heiligabend hatte; damit er die BTLs abliefere mit seinem schnellen Auto, das da war ein Toyota.
Und als er Düsseldorf erreichte kam die Zeit, dass er die BTLs verladen sollte.
Und er verlud die BTLs und wickelte sie in Geschenkpapier und legte sie in seinen Kofferraum, und oben drauf einen hässlichen Karl-Kombatmage-Strickpullover, denn man sollte die BTLs nicht finden.

Doch es waren Grenzschutzpolizisten auf der Grenze vom Rhein-Ruhr-Plex zum Kirchenstaat Westfalen, die hüteten des Nachts dort die Schlagbäume; und sie tranken heimlich Feuerzangenbowle, denn sie wären lieber zu Hause bei ihren Familien gewesen.
Und der Kurier sah das und fuhr zu ihnen, und die Maria-Mercurial-Playlist aus seiner Musikanlage ertönte um sie, und sie wunderten sich sehr.
Und der Kurier sprach: \enquote{Wundert euch nicht! Seht, ich bringe euch Spekulatius und Zimtsterne als Weihnachtsgeschenk, und außerdem eine Summe von 500 Nuyen, damit ihr am heiligen Abend ein Auge zudrückt; denn ihr habt mich nicht gesehen!}
Und alsbald war bei dem Auto des Kuriers die Menge der beschwipsten Grenzschutzpolizisten, die lobten die Weihnachtsplätzchen und sprachen:
\enquote{Dank sei dem, der diese Plätzchen gebracht hat; und gesegnet sei das Weihnachtsfest dieses Halunken!}
Und als die Polizisten den Schlagbaum öffneten, sprach der Kurier zu sich:
\enquote{Wenn ich nun nach Berlin fahre, wird keiner die Geschichte glauben, die hier geschehen ist!}

Und er fuhr eilends und fand eine verborgene Route über dunkle Feldwege und erreichte Berlin.
Als er die Stadt aber erreicht hatte, lieferte er die BTL-Chips ab, und erhielt die vereinbarte Bezahlung und ein großzügiges Trinkgeld.
Den Karl-Kombatmage-Strickpullover aber behielt er, denn der hatte keine Blutflecken und war sehr kuschelig.
Und die, denen er von seiner Fahrt erzählte, wunderten sich sehr über das, was ihnen der Kurier berichtete.
\Verfasser[SR 5]{Franca}

\begin{termine}
% Put dates here:
  \item Monatliches Treffen: 16.01.2019, 19~Uhr
  \item Librarium Freizeit: 02.01.2020 - 06.01.2020
\end{termine}

\subsection{Weihnachtsansprache }
Liebe Mitbürgerinnen und Mitbürger,\\
Liebe Mitarbeiterinnen und Mitarbeiter,\\
Liebe Angestellte und Gefangene,

Jedes Jahr trete ich vor Sie, um ein paar Sätze über den Stand unseres
Konglomerats zu verlieren. Auch dieses Mal möchte ich mich bei Ihnen
allen für Ihr Mitwirken bedanken. Unsere Ergebnisse wären weder ohne den
tatkräftige Einsatz von tausenden Mitarbeiterinnen und Mitarbeitern noch
ohne die exzellente Führung unseres Vorstands möglich gewesen.

Neben diesen warmen Worten muss ich aber leider auch ein paar kühlere
verlieren. Der Druck der Konkurrenz ist hoch und auch wir müssen uns den
Regeln der Welt fügen. Gestern wurde das vergangene Geschäftsjahr
abgeschlossen und die letzten Quartalszahlen sind fertig. Leider sind
diese nicht so ausgefallen, wie wir uns das vorgestellt haben.
Wir konnten in diesem Jahr den Absatz um 18\% und unseren
Gewinn nur um 6,8\% steigern. Damit bleiben wir weit hinter den
Erwartungen der Aktionäre unserer Vorstände und natürlich auch hinter
Ihren Erwartungen zurück.
Sowohl bei der Markteinführung der neuen ITEMA-17 supraleitenden Motoren
als auch der neuen Cyberplus Produktpalette gibt es nachhaltiges Verbesserungspotential.

Lasst uns in der kalten Jahreszeit zusammenrücken und etwas Wärme
teilen. Lasst uns alle unseren Gürtel ein wenig enger schnüren und weiter auf
eine glorreiche Zukunft hinarbeiten. Ich weiß, was ich verlange ist
viel und nicht selbstverständlich, aber um unsere gemeinsame Zukunft zu
sichern, müssen wir alle Opfer bringen. Die Konzernleitung musste auf
Grund der schlechten Zahlen die Entscheidung treffen, dass in diesem
Jahr keine Weihnachtsgelder ausgezahlt werden und betriebsbedingte Entlassungen 
vorgenommen werden müssen. Viele bis zum Jahresende befrisstete Stellen können leider nicht verlängert werden.
Die Betroffenen werden rechtzeitig bis zum Jahresende darüber informiert.

Doch seien Sie versichert. Auch wir, die Konzernleitung, wird sich den
neuen Gegebenheiten anpassen müssen.
Mit schwerem Herzen muss ich Ihnen deshalb heute verkünden, dass meine Vorstandskollegen ihren Posten räumen werden, um
Platz für neue und frische Ideen in diesen schweren Zeiten zu machen.
Ich bedaure diesen Schritt und werde noch lange auf diese von Vertrauen
und Herzlichkeit geprägte Zeit zurückblicken.
Auch Ihnen, liebe Mitbürgerinnen und Mitbürger, liebe Mitarbeiterinnen
und Mitarbeiter, liebe Angestellte und Gefangene wünsche ich noch
gesegnete Feiertage und eine erfolgreiche Zukunft.
\verfasser[SR 5]{Yoann}

\impressum

%\newpage
%
%\subsection{Lipsum}
%Hier könnte ihre WErbung stehen. Dieser Platzhalter wird präsentiert von Biobrause. BioBrause, jetzt neu in der Geschmacksrichtung Rosenkohl-Nutella!
%
%\begin{termine}
%% Put dates here:
%  \item Monatliches Treffen: 16.01.2019, 19~Uhr
%  \item Librarium Freizeit: 02.01.2020 - 06.01.2020
%\end{termine}

\end{document}
%%%%%%%%%%%%%%%%%%%%%%%%%%%%%%%%% END DOCUMENT %%%%%%%%%%%%%%%%%%%%%%%%%%%%%%%%%
