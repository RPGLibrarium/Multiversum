% !TeX TS-program = xelatex
% !TEX root = main.tex

% How to:
%


% Load class with all definitions. Do not remove this line.
% Options will be passed to Memoir
\documentclass[final]{multiversum}
%

%%%
% Set those variables

% Authors of the document.
% e.g. Max Mustermann, Erika Musterfrau
\multiauthor{Hanna, Yoann}

% Date of release.
% e.g. 31.12.2074
\multidate{April 2020}

% Number of release, no leading zeros.
% e.g. 15
\multiausgabe{21}

% Losung
% e.g. Die Kuh lief um den Teig.
\multilosung{}


% Logo
% Use a different logo. Defaults to Ueberschrift.svg
%\multilogo{Ueberschrift_xmas}

%
%%%

\newcommand\fnref[2]{\href{#1}{#2}\footnote{\url{#1}}}

%%%%%%%%%%%%%%%%%%%%%%%%%%%%%%%%% DOCUMENT %%%%%%%%%%%%%%%%%%%%%%%%%%%%%%%%%
\begin{document}

\makemultititle
%

% PUT BODY HERE
\section{Was bisher geschah...}

\subsection{Digitale, nicht beschlussfähige MV - wait, what?}
Inzwischen haben wir uns an unser neues, eingeigeltes Leben gewöhnt und die Fähigkeit verloren, anderen in die Augen zu schauen.
Das liegt an der Schwierigkeit, gleichzeitig in die Webcam und auf den Bildschirm zu schauen.
Entsprechend gerne würden wir eine echte Mitgliederversammlung besuchen und uns mal wieder ein Blickduell liefern. 
Leider ist das noch nicht möglich, daher war die MV am 27. Februar 2021 auch digital. 
Unpraktisch, da wir nur in einer persönlichen Zusammenkunft beschlussfähig sind.
Warum also überhaupt eine MV?
Die Antwort haben die Anwesenden recht schnell verstanden.
Wir konnten so schon einmal Einblick in alles wichtige bekommen:
\begin{itemize}
\item Trotz aller Widerigkeiten konnten 2020 11 (digitale) Monatstreffen mit zunehmender Professionalität stattfinden.
Was da passiert ist und ob ihr etwas verpasst habt erfahrt ihr in unserer \fnref{https://rpg-librarium.de/veranstaltungen/}{Veranstaltungsliste}.
\item Um den Mitgliedern weiter den Zugang zu ermöglichen, steht der Bücherschrank bis auf weiteres bei Franca. 
Angeblich wird er von einem Drachen bewacht, aber auf Anfrage erkämpft euch Franca ein Buch. 
Die Adresse der Drachenhöhle gibt es über eine Mail an \href{mailto:vorstand@rpg-librarium.de}{vorstand@rpg-librarium.de}.
\item Auch wenn der Jahresabschluss nicht beschlossen wurde, konnte der Vorstand Einblicke geben. 
Bei Interesse könnt ihr diese Infos auch vom \href{mailto:vorstand@rpg-librarium.de}{Vorstand} bekommen.
\item Da die Vorstandswahl verschoben werden musste, gab es einige Meinungsbilder - was wollen wir dieses Jahr tun? 
Sollen wir eine Freizeit 2022 in Erwägung ziehen? 
Unser bisheriger Vorstand gibt sich alle Mühe, bei nächster Gelegenheit nicht abgewählt zu werden. 
Wenn ihr nicht dabei wart und etwas ergänzen wollt, meldet euch beim \href{mailto:vorstand@rpg-librarium.de}{Vorstand}!
\item Natürlich haben wir auch Bücher- und diesmal Landkartenwünsche gesammelt.
Die könnt ihr natürlich jederzeit äußern. Schreib auch hierfür an den \href{mailto:vorstand@rpg-librarium.de}{Vorstand}.
\end{itemize}
Irgendwann, wenn wir uns wieder aus unseren Höhlen wagen können, bekommen wir wieder eine richtige MV mit Blackj\dots äh, nein, beschlussfähig natürlich. 
Bis dahin findet ihr alle wichtigen Updates auf der \fnref{https://rpg-librarium.de/}{Webseite} und im \fnref{https://lists.rpg-librarium.de/postorius/lists/newsletter.lists.rpg-librarium.de/}{Newsletter}.
\verfasser{Hanna}

\section{An einem anderen Ort}

\subsection{Tavernengespräch, gehört in Boran}
Es fing alles in Mendena an.
Blödes Dämonennest, befreit von der Kaiserin am Arsch!
Da hätte ich nie anwerben sollen.
Es war nun mal der einzige freie Hafen am gesamten Perlenmeer.
Oder der blutigen See, oder was auch immer.
Zwanzig Jahre habe ich gewartet, da hat der Mocha mir das Bein geraubt.

Was, der Mocha?
Ein Hai natürlich!
Aber nicht irgendein Hai, nein!
Der Mocha ist der gigantischste, dämonenverseuchteste Ifirnshai südlich des scheiß Eismeers.
Seine Finne ist vom brillantesten weiß, jeder Idiot würde ihn erkennen und kehrt machen.
Nicht ich, natürlich.
Ich hätte ihn schon einmal fast gehabt, vor zwanzig Jahren, meine Harpune steckte schon!
Aber das Vieh hat Kraft. 
Hat mich vom Jagdboot gerissen, mitsammt der Leine, und ist mit meinem vermaledeitem Bein abgehauen.
Musste mir ein neues Schnitzen, aus dem Kiefer eines kleineren Ifirnhais.
Aber damals habe ich geschworen, dass ich der Untergang des Viehs sein werde.
Und das werde ich einhalten, bei Efferd!

Ich gehe jede Wette ein, dass ich ihn schon gefunden hätte, wenn die verdammten Thorwaler nicht angeheuert hätten.
Barbaren, Götzenanbeter sind das.
Völlig wahnsinnig.
Habe sie in Mendena aufgesammelt.
Niemand wollte die blutige See mit mir bereisen, egal wie viel Ruhm und Geld es versprach.
Ich habe gutes Geld geboten, die letzten zwanzig Jahre habe ich meinen Reichtum gemehrt, um meine Sturmvogel gut instand zu halten und diese eine Fangfahrt vorzubereiten.
Dann kam diese komische Gruppe.
Mein Geweihter Pequod fand sie von Anfang an merkwürdig, aber niemand dachte, dass sie das verdammte Schiff versenken.
Der Anführer war ein Thorwaler und der hatte auch ein paar seiner eigenen Leute dabei.
Dazu gab es zwei oder drei Magier, zwei davon waren Zwerge.
Verrücktes Volk, haben die ganze Zeit über die Rehling gereihert.
Außerdem noch Mittelländer und Inselvolk, eine komische Truppe.
Söldner aus ganz Aventurien.
Ich sage euch, nicht mal die Thorwaler waren alle Seefahrer.
Einer hat sich beim Ausfahren im Hafen an der Takellage K.O. geschlagen, hat mein Steuermann mir berichtet.

Sie wollten keine Haie jagen, diese Gruppe, sondern Seeschlangen.
Welcher Mensch bei Verstand würde sich einen Kampf mit einer Seeschlange auf dem offenen Meer wünschen?
Hab in all meinen Jahren auf der See nur drei Male welche gesehen, immer von der Ferne, und von den schlimmen Dämonengebieten haben wir uns fern gehalten, das war abgemacht.
Im Gegenzug sollten die Barbaren für mich Haie jagen, ich dachte wir treffen auf keines der Schlangenviehcher und damit schlägt sich Profit heraus.

Ob deren Götzengott das eingerichtet hat oder die Truppe einfach Pech gebracht hat, weiß ich nicht.
Ich weiß nur eines.
Noch nie zuvor ist mir an der Firunsseite von Maraskan eine Seeschlange begegnet, bis die Unglücksvögel angeheuert haben.
Unsere Ausbeute an Haien war nur minimal, da stießen wir auf gleich 2 gigantische Schlange, umeinander geschlungen.
Wenn ich nicht wüsste, dass die Viehcher nur in den Dämonenarchen gebrütet werden, würde ich schwören dass die sich gepaart haben!
Und die Thorwaler sind voll in den Kampf gesprungen.
Wahnsinnige Bastarde!
Deren Anführer, Foggwulf nennen sie ihn, der schien Erfahrung zu haben. 
Hat den ganzen wahnsinnigen Kampf geleitet.
Die Schlangen haben sich um unser Schiff geschlungen und versucht, uns herunterzuziehen oder bei dem Versuch auseinanderzubrechen.
Das schwöre ich bei meinem verbliebenen Bein!
Mit den Bordgeschützen konnte man nichts gegen sie ausrichten, aber die hatten diesen mittelreichischen Bogenschützen dabei, der hat keine Gelegenheit ausgelassen die empfindlichen Stellen zu treffen.
Wie von magischer Hand geleitet hat er Pfeil um Pfeil in die Augen der Schlangen geschossen, beinahe gruselig, jeder Schuss ein Treffer.
Wisst ihr Volk, wie man eine Seeschlange mit vollen Eiern noch mehr reizt?
Indem man sie verdammt noch mal blendet!
Den Haupmast haben sie mit abgerissen und ein zweites Segel ist auch fast draufgegangen!
Der Rumpf ist aufgerissen und der gesamte Laderaum vollgelaufen, bis hoch zum Mannschaftsraum!
Der Hund von dem einen Thorwaler saß noch winselnd auf den höchsten Stapeln und wäre fast abgesoffen.
Hätten die verdient.
Wenn der eine Magier nicht irgendwas mit dem Wasser gemacht hätte, um das Leck zu stopfen, wären wir einfach abgesoffen.
Die Hälfte meiner Rehling ist ins Meer verschwunden!
Das war bestes Meeresfischbein, alles selbst gefangen und von meinem Zimmermann zugeschnitzt.
Armer Herbert, der ist auch draufgegangen. 
Ist mit zehn anderen über Bord gegangen und im Rachen der Schlangen gelandet.
Irgendwann konnten wir den blinden Schlangen entkommen, aber es war knapp.

Da waren wir nun, in der Mitte der blutigen See, Leck geschlagen mit dem halben Schiff voll Wasser, mit nur zwei Nebensegeln einigermaßen intakt.
Der einzige sichere Hafen hier in Boran, tagesreisen entfernt.
Efferd muss nun doch Mitleid mit uns gehabt haben, wir haben es noch geschafft.
Meine schöne Sturmvogel liegt im Trockendeck und ob wir vor dem nächsten Winter wieder in See stechen können, wissen nur die Götter.
Nichts für ungut, aber bis dahin stecken wir in dieser \textit{wunderschönen} Stadt fest.
Verdammt.
Bis dahin hat der Mocha seine Wanderung ins Eismeer begonnen und er entwischt mir wieder.
Wisst ihr, was ich denke?
Der Dämon, der den Mocha geboren hat, hat mir diese Thorwaler geschickt.
Das Vieh hat eine schützende Hand über sich, und die will mich zerquetschen.
Aber da kennt ihr Kapitän Bacha schlecht!
Nächstes Jahr kriege ich den Mocha, und dann kenne ich keine Gnade!
\verfasser[DSA 4.1]{Hanna}


\begin{termine}
% Put dates here:
\item Monatstreffen: 16.04.21, 19 Uhr

\end{termine}
\impressum

\end{document}
%%%%%%%%%%%%%%%%%%%%%%%%%%%%%%%%% END DOCUMENT %%%%%%%%%%%%%%%%%%%%%%%%%%%%%%%%%
