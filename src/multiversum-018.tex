% !TeX TS-program = xelatex
% !TEX root = main.tex

% How to:
%
 

% Load class with all definitions. Do not remove this line.
% Options will be passed to Memoir
\documentclass[final]{multiversum}
%

\usepackage{enumitem}
%\usepackage[markup=nocolor,deletedmarkup=xout]{changes}
\usepackage{cancel}


%%%
% Set those variables

% Authors of the document.
% e.g. Max Mustermann, Erika Musterfrau
\multiauthor{Hanja, Konstantin, Franca, Richard}

% Date of release.
% e.g. 31.12.2074
\multidate{Januar 2020}

% Number of release, no leading zeros.
% e.g. 15
\multiausgabe{18}

% Losung
% e.g. Die Kuh lief um den Teig.
\multilosung{Schwarze Magie brennt schlechter!}


% Logo
% Use a different logo. Defaults to Ueberschrift.svg
%\multilogo{Ueberschrift_xmas}

%
%%%

%%%%%%%%%%%%%%%%%%%%%%%%%%%%%%%%% DOCUMENT %%%%%%%%%%%%%%%%%%%%%%%%%%%%%%%%%
\begin{document}

\makemultititle
%

% PUT BODY HERE
\section{Was bisher geschah...}

\subsection{Freizeit}
Vor nicht ganz einem Monde, so erzählt man sich, begab es sich zur Kirchen bei Sieg.
Die altehrwürde Freusburg, gestürmt wurde sie, von einer wilden Horde.
Magier, Trolle, Zwerge, blecherne Wesen, aber auch leicht bekleidete Gestalten und sogar Tiere aus Textilfaser wurden gesichtet!

Über 5 Tage hinweg belagerten sie den Musikturm und plünderten das Buffet!
Erzählt wurde von fremden Welten, dabei geschmissen mit komischen Steinen.
Nach dem letzten Mahl sind sie sodann entschwunden, so schnell, wie sie gekommen warn.
Seither wurden sie nicht mehr erblickt, in jeden fernen Tal.
\Verfasser{Richard}

\subsection{Spaß mit Zahlen}
Von welchem System ist der Buchvorrat am größten?
Welches Buch das wertvollste?
Wie viele Werke sind neu hinzugekommen?
Was wollen wir noch haben?

Antworten auf diese Fragen gab es beim Treffen im Januar.
Neben einem Jahresrückblick und ansprechenden Visualisierungen vieler spannender Zahlen wurde eine Wunschliste fürs neue Jahr geschrieben.

Noch mehr Zahlen (in Form von Jahresabschluss und Finanzplan) erwarten euch auch im Februar, denn dann ist es mal wieder Zeit für eine Mitgliederversammlung des Librariums.
Kommt vorbei, stürzt die Aristokratie, startet eine Revolution … oder trinkt einfach nur einen Kakao.\verfasser{Richard}


\section{An einem anderen Ort}
\setsubsubsecheadstyle{\centering\bfseries}

\subsection{Schwarzes Brett von Bad Pfaff}
\begin{center}
    \textbf{Vermisst}:
    Emeli Arzât
\end{center}
\vspace{-1.5em}
\noindent
Junge, rothaarige Menschenfrau mit grünen Augen.
Zuletzt am Abend des zweiten Helmstag von Tarsak gesichtet.
Bei Hinweisen im Stadthaus melden.
50 GM für Hinweise, die zur Auffindung helfen.
\Verfasser[D\&D5]{Hanja}


%\subsection{Ausschnitt aus dem Wochenblatt von Blaumar}
\subsection{Die Werwölfe erfolgreich vertrieben!}

Die Stadtwache sowie Freiwillige haben in der Nacht des zweiten Dreitag von Tarsak die Werwolfsplage beseitig. 
Ein mutiger Kämpfer und Teil der Crew der Sturmspitze wird weiterhin vermisst. 
Es wird vermutet, dass sie ihr Leben im Kampf lassen mussten.
Eine Ehrung der edlen Retter wird am zweiten Fünftag von Tarsak auf dem Marktplatz stattfinden. Für die Seele des verlorenen Kriegers wird im Anschluss eine Messe gehalten.
\zeitung{Ausschnitt aus dem Wochenblatt von Blaumar}
\verfasser[D\&D5]{Hanja}


\subsection{Produktrückruf Besitzurkunde}
Die zur Wintersonnenwende von dem Schriftmeister Pairo (nom. script.) ausgestellte Urkunde wird zurückgerufen.
Die fehlerhafte Ausstellung ist dem vorzüglichen Wein der Katakombentaverne geschuldet.

Entgegen der Behauptung der Urkunde haben die Besitzenden der Urkunde nicht \textit{notwendigerweise} einen Besitzanspruch an allen Pflöcken der Welt.
Die Angabe, dass sie darüber hinaus Anrecht auf vier weitere Pflöcke hätten, ist ebenfalls fehlerhaft.
Der Autor maßt sich keine Aussage über die Rechtsgeschäfte der als \enquote{die Rauferei} bekannten Gruppe an.
Ihm ist insbesondere \textit{nicht} bekannt, dass diese Gruppe unehrenhaft handeln würde, Essen oder Kleidung gestohlen hätte oder einen illegalen Kampfring betreiben würde.
\verfasser[LARP]{Konstantin}


\subsection{Notiz am Schwarzen Brett von Waldblick}
\begin{center}
    \textbf{Thif}
\end{center}
\vspace{-1.5em}
\noindent
Ich bin weiter Richtung Westen. \\
Wir treffen uns in Thordesheim am ersten Vierttag von Mirtul.

Thade
\verfasser[D\&D5]{Hanja}


\DeclareRobustCommand{\sxout}[1]{\texorpdfstring{\xout{#1}}{#1}}
\subsection{Schwarzes Brett von \xcancel{Herzogtum} Lindstein}
Öffentliche Hinrichtungen dritten Ersttag von Tarsak. Feier mit Wein und gegrilltem Wolf.
Hinweise auf lebende Adelseingebildetheiten werden mit Wein und Ehre entlohnt.
\verfasser[D\&D5]{Hanja}


\subsection{Büromaterialbestellung -- Eilt!}
An: Zentrales Ressourcenmanagement \\
Von: Miesermeier, Zweigstelle Kirchen (Sieg) \\

\noindent
Liebe Kollegen der INSPECTRES Zentrale,

während der laufenden Einarbeitungszeit mussten wir leider feststellen, dass noch Ressourcen vor Ort benötigt werden.
Diese ergeben sich sowohl aus dem letzten Einsatz, als auch aus dem generellen Materialbedarf.
Daher beantrage ich hiermit die folgenden Ressourcen zu dienstlichen Zwecken:

\begin{enumerate}[leftmargin=*, itemsep=1ex]
  \item Ein Schreibtisch, bevorzugt Buche. \\
    Art der Beschaffung: Ersatz. (Für den defekten Tisch, der mir zur Einstellung zugeteilt wurde.) \\
    Begründung:
      Der vorhandene Schriebtisch kann nicht mehr als solcher bezeichnet werden.
      Er ist, vermutlich bedingt durch Materialfehler, schon am ersten Tag bei geringster Belastung zusammengebrochen.

  \item Sonderentsorgungscontainer. \\
    Art der Beschaffung: Temporär. (Logistik.) \\% zu Transport- bzw. Entsorgungszwecken. \\
    Begründung:
      Überreste einer Entität (vogelähnlich, Größenklasse 1-2m, mittels Axt zerstückelt) müssen zeitnah vom Burghof entfernt werden.
      Vermutlich übernatürlich, daher für die Standard-Abfallentsorgung ungeeignet.
      Kann Reste von Arzneiprodukten enthalten.
      (Siehe auch: Einsatzbericht vom 2020-01-02.)

  \item Personal. \\
    Art der Beschaffung: Ersatz. \\
    Begründung:
      Im Zuge des letzten Einsatzes ist Z4 gegangen und bisher nicht wieder zum Dienst erschienen.
      Es ist nicht mehr mit einer zeitnahmen Rückkehr zu rechnen.
      Wir benötigen personelle Verstärkung.
      (Diesmal bevorzugt nicht in kommende Fäll verwickelt.)

  \item Lärmschutzdämmmaterial. \\
    Art der Beschaffung: Neubeschaffung, bauliche Maßnahme. \\
    Begründung:
      Unter der akuten Lärmbelastung durch die Musik des Kollegens Chopper lässt sich hier nicht mehr produktiv arbeiten.
      Eine Bauliche Maßnahme scheint hier die sinnvollste Lösung zu sein.

  \item KFZ-Radio. \\
    Art der Beschaffung: Ersatz. \\
    Begründng:
      Das vorhandene Radio des Dienstpanzers kam in Kontakt mit einer Rohrzange und wurde dabei tragischerweise beschädigt.
      Da es sich nun nicht mehr abschalten lässt und die Nachbarn sich beschweren, ist schnellstmöglich Ersatz erforderlich.
      (Außerdem weigert sich der Kollege ohne Musik zu fahren, daher ist ein Ausbau ohne Ersatz keine Option.)
\end{enumerate}

\noindent
Mit besten Grüßen, \\
Miesermeier
\verfasser[Inspectres]{Richard}


\subsection{Rezept des Tages: Grillratte „Barovia“}

Das perfekte Gericht für neblig-kalte Tage!

Was gibt es Schöneres, als bei schlechtem Wetter gemütlich mit Freunden in einer leerstehenden Ruine zu sitzen und über einem wohlig prasselnden Feuer selbstgefangene Ratten zu rösten? Barovianische Grillratten sind das perfekte Soulfood für Abenteurer und schmecken einfach jedem!

Schritt 1: \textit{Die richtige Lokalität}
Größere Schwärme von Ratten finden sich oft in leerstehenden Gebäuden. Durch beherztes Klopfen an der Tür kann getestet werden, ob ein Haus tatsächlich unbewohnt ist. Antwortet niemand, kann die Tür unbesorgt mit einer Axt eingeschlagen werden. Wenn das Haus von Ratten besiedeln ist, werden diese zeitnah attackieren – jetzt ist etwas Geduld gefragt.
Im Gegensatz zu Pilzen sind grundsätzlich alle Arten von Ratte essbar. (Allerdings sollte – anders als bei Pilzen – der Kopf nicht gegessen werden.)

Schritt 2: \textit{Gemeinsame Rattenjagd}
Hier kann die Axt aus Schritt 1 (Achtung! Verbleibende Holz- und Lacksplitter von der Tür vorher von der Axt entfernen!) oder das präferierte Küchenwerkzeug (Spektrale Waffen, Fußtritte...) genutzt werden. Ungeeignet zur Rattenjagd sind Schadensflüche mit nekrotischem Aspekt, denn dadurch wird das Rattenfleisch ungenießbar.
Zum Vermengen der Ratten eignet sich eine magische Donnerwoge. Dabei den Gehörschutz nicht vergessen und vorsichtig sein, dass Umstehende nicht verletzt werden. Dieser Zauber sollte nicht in allzu baufälligen Häusern verwendet werden; es droht Einsturzgefahr!

Schritt 3: \textit{Zubereitung der Ratten über dem Lagerfeuer}
Zertrümmerte Bodendielen und alte Möbel sind die perfekte Grundlage für ein Indoor-Lagerfeuer. Die Hauswand sollte in Folge von Schritt 2 genug Löcher haben, sodass es zu Durchzug kommt und potentieller Qualm sich nicht staut. Im Gebäude ein Lagerfeuer entzünden - auf magische oder mundane Weise, Hauptsache, es ist heiß. Die Ratten auf einen Stock spießen und gleichmäßig über dem Feuer rösten. Sobald sich erste Verkohlungen zeigen, sind die Ratten fertig.

Tipp: Der Connaisseur trinkt zur Barovianischen Grillratte einen schweren Rotwein, gerne auch in größeren Mengen, und verzichtet dankend auf die Ratte.

\verfasser[D\&D 5]{Franca}


\begin{termine}
% Put dates here:
  \item Monatliches Treffen: 16.05.2020, 19~Uhr
\end{termine}
\impressum

\end{document}
%%%%%%%%%%%%%%%%%%%%%%%%%%%%%%%%% END DOCUMENT %%%%%%%%%%%%%%%%%%%%%%%%%%%%%%%%%
