\documentclass[a4paper,10pt]{article}
\usepackage[utf8]{inputenc}
\usepackage[english,ngerman]{babel}

\title{Readme: Wie man das Flugblatt-Layout benutzt. \\
       Konventionen, ein \LaTeX-Kurztutorial und nützliche Makros.}
\author{Hanna Franzen}
\date{\today}

\begin{document}
\maketitle

Du hast dich dazu entschieden an unserem wundervollen Flugblatt mitzuarbeiten, juchhu!
Hier sind die ersten Schritte zur Nutzung unseres Layouts:

\section{Installieren und einrichten von \LaTeX}

\begin{enumerate}
   \item Installiere \LaTeX. Je nachdem welches Betriebssystem du benutzt kann das unterschiedlich aussehen.
         Dazu findet man im Internet einige sehr gute Anleitungen 
         (gebe zum Beispiel ``\textit{LaTeX installieren Windows 10}'' bei Google ein und wähle eine der Anleitungen).
      
         \textbf{Achtung bei Windows}: Damit es später nicht zu Problemen kommt solltest du das Programm MiKTeX 
         installieren \textbf{bevor} du einen Editor installierst. Darauf werden dich wahrscheinlich einige der gefundenen 
         Anleitungen auch hinweisen.
      
   \item Installiere einen \LaTeX-Editor. Das ist z. B. Kile für Linuxsysteme oder TexMaker, das für verschiedene
         Systeme verfügbar ist. Falls du noch keinen Lieblingseditor hast oder noch nie mit \LaTeX gearbeitet hast,
         probiere doch einen von diesen Beiden aus. Geeignete Editoren habenden Vorteil, dass sie den Befehl
         ``\textit{PDFLaTeX}'' können (siehe nächsten Schritt).
         Natürlich kannst du für kleine Änderungen auch einen vom System mitgebrachten Editor wie den \textit{Editor}
         auf Windows oder \textit{Gedit} auf Ubuntu benutzen. Dann beachte aber bitte trotzdem unsere Konventionen im
         nächsten Abschnitt.
        
   \item Öffne mit deinem Editor das Dokument ``main.tex'' des Flugblatts. An diesem Dokument sollte man nur etwas ändern,
         wenn man genau weiß was man da tut. Dein Editor sollte nun den Befehl ``\textit{PDFLaTeX}'' können. 
         Damit wird das PDF erstellt und aktualisiert, falls etwas an den einzelnen Abschnitten geändert wurde. 
         Wenn du also eine Änderung gemacht hast solltest du damit immer ausprobieren, ob sie so aussieht wie du es 
         beabsichtigt hast.
\end{enumerate}

\section{Konventionen}

Damit die Texte übersichtlich bleiben beachte bitte die folgenden Hinweise:
\begin{itemize}
   \item Die Zeilen in \textit{.tex}-Dokumenten sollten nicht länger als 125 Zeichen sein. Die meisten \LaTeX-Editoren
         geben das laufend an, z.B. steht bei Kile in der rechten unteren Ecke eine Angabe von der Form
         ``\textit{Line: 41 Col: 33}''. Das bedeutet, man ist in Zeile 41 beim 33. Zeichen. Da man in \LaTeX einen
         Absatz theoretisch in einer einzigen Zeile schreiben kann wird der Text schnell unlesbar, obwohl er im PDF
         gut aussieht. Falls man mal schnell einen Fehler korrigieren möchte ist das unpraktisch.
         Also: Maiximal 125 Zeichen pro Zeile.
         
         \textbf{Tipp:} Um bei größeren Änderungen nicht dauernd ändern zu müssen wo die Zeile endet kann man
         z.B. nach jedem Satz oder sogar Nebensatz einen Zeilenumbruch machen.
\end{itemize}

\section{Kurzeinleitung in \LaTeX}

\section{Nützliche Befehle}


\end{document}
